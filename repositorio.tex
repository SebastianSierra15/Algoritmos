\documentclass[10pt,landscape,twocolumn,a4paper,notitlepage]{article}

\usepackage{hyperref}
\usepackage[spanish, activeacute]{babel}
\usepackage[utf8]{inputenc}
\usepackage{fancyhdr}
\usepackage{lastpage}
\usepackage{listings}
\usepackage{amssymb}
\usepackage[usenames,dvipsnames]{color}
\usepackage{graphicx}
\usepackage{wrapfig}
\usepackage{amsmath}
\usepackage{makeidx}
\usepackage{verbatim}
\usepackage{color}
\usepackage{geometry}
\usepackage{multicol}
\usepackage{multirow}
\usepackage{supertabular}
\usepackage{booktabs}
\geometry{verbose,landscape,letterpaper,tmargin=2cm,bmargin=2cm,lmargin=1cm,rmargin=1cm}


%%%%%%%%%%%%%%%%%%%%%importar codigo desde archivos cpp
\lstloadlanguages{C++}
\lstnewenvironment{code}
	{%\lstset{	numbers=none, frame=lines, basicstyle=\small\ttfamily, }%
	 \csname lst@SetFirstLabel\endcsname}
	{\csname lst@SaveFirstLabel\endcsname}
\lstset{% general command to set parameter(s)
	language=C++, basicstyle=\small\ttfamily, keywordstyle=\slshape,
	emph=[1]{tipo,usa}, emphstyle={[1]\sffamily\bfseries},
	morekeywords={tint,forn,forsn},
	basewidth={0.47em,0.40em},
	columns=fixed, fontadjust, resetmargins, xrightmargin=5pt, xleftmargin=15pt,
	flexiblecolumns=false, tabsize=2, breaklines,	breakatwhitespace=false, extendedchars=true,
	numbers=left, numberstyle=\tiny, stepnumber=1, numbersep=9pt,
	frame=l, framesep=3pt,
    basicstyle=\ttfamily,
    keywordstyle=\color{blue}\ttfamily,
    stringstyle=\color{magenta}\ttfamily,
    commentstyle=\color{RedOrange}\ttfamily,
    morecomment=[l][\color{OliveGreen}]{\#}
}

\lstdefinestyle{C++}{
	language=C++, basicstyle=\small\ttfamily, keywordstyle=\slshape,
	emph=[1]{tipo,usa,tipo2}, emphstyle={[1]\sffamily\bfseries},
	morekeywords={tint,forn,forsn},
	basewidth={0.47em,0.40em},
	columns=fixed, fontadjust, resetmargins, xrightmargin=5pt, xleftmargin=15pt,
	flexiblecolumns=false, tabsize=2, breaklines,	breakatwhitespace=false, extendedchars=true,
	numbers=left, numberstyle=\tiny, stepnumber=1, numbersep=9pt,
	frame=l, framesep=3pt,
    basicstyle=\ttfamily,
    keywordstyle=\color{blue}\ttfamily,
    stringstyle=\color{magenta}\ttfamily,
    commentstyle=\color{RedOrange}\ttfamily,
    morecomment=[l][\color{OliveGreen}]{\#}
}
 
%%% Macros
\def\nbtitle#1{\begin{Large}\begin{center}\textbf{#1}\end{center}\end{Large}}
\def\nbsection#1{\section{#1}}
\def\nbsubsection#1{\subsection{#1}}
\def\nbcoment#1{\begin{small}\textbf{#1}\end{small}}
\newcommand{\comb}[2]{\left( \begin{array}{c} #1 \\ #2 \end{array}\right)}
\def\complexity#1{\texorpdfstring{$\mathcal{O}(#1)$}{O(#1)}}
 \newcommand\cppfile[2][]{
\lstinputlisting[style=C++,linerange={#1}]{#2}
}

\begin{document}
	
	\title{Repositorio en c++}
	\author{Universidad de la amazonia}
	\maketitle
	
	\tableofcontents\newpage
	
		\section{Estructuras de datos}
			\subsection{tablas aditivas}
			Construccion O(n)
			\cppfile[16-28]{estructuras_de_datos/tablas_aditivas.cpp}
			\subsection{disjoint set union find}
			Construccion O(n)\\asocia elementos en conjuntos de arboles.
			\cppfile[6-53]{estructuras_de_datos/union_find.cpp}
			\subsection{union find con compresion de caminos}
			asocia elementos de manera simple.
			\cppfile[7-40]{estructuras_de_datos/union_find-compresion_de_caminos.cpp}
			\subsection{segment tree}
			Ejemplo de RMQ (Range Minium Query)\\
			Contruccion O(n)\\Consulta O(log n)\\Update O(log n)
			\cppfile[8-70]{estructuras_de_datos/segment_tree.cpp}
			
		\section{Grafos}
			\subsection{Dijkstra}
			Ruta minima
			O((n + m)log n)
			\cppfile[6-38]{grafos/dijkstra_con_priority_queue.cpp}
			\subsection{Bellman-Ford}
			Ruta minima con pesos negativos
			O($n^{2}$)
			\cppfile[5-39]{grafos/bellman-Ford.cpp}
			\subsection{Floyd Warshall}
			Ruta minima de toda la matriz, recomendable si n $\leq$ 100\\	
			O($n^{3}$)
			\cppfile[4-19]{grafos/floyd.cpp}
			\subsection{Kosaraju}
			Componentes fuertemente conexas grafos si y no dirigidos\\
			O(2(n + m))
			\cppfile[7-45]{grafos/kosaraju.cpp}
			\subsection{Kruskal}
			Arbol generador minimo, se necesita de un union-find\\
			O(m log n), sin contar el ordenamiento.
			\cppfile[43-67]{grafos/kruskal.cpp}
			\subsection{Topological sort}
			O(m + n)
			\cppfile[7-31]{grafos/topological_sort_para_grafos_ciclicos.cpp}
			
		\section{Matematicas}
			\subsection{MCD y MCM}
			Maximo comun divisor(MCD) y minimo comun multiplo(MCM)
			\cppfile[5-11]{matematicas/MCD_y_MCM.cpp}
			\subsection{Exponenciacion binaria}
			O(log n)
			\cppfile[4-15]{matematicas/exponenciacion_binaria.cpp}
			\subsection{Multiplicacion modular}
			Encuentra (a*b) mod c, la operacion puede generar overflow
				si se realiza directamente, el metodo mulmod evita el overflow usando un
				ciclo, pero se puede usar el tipo de dato int128 de c++11 para poder calcular
				de manera directa, pero el int128 no se puede leer o imprimir directamente.
			\cppfile[5-26]{matematicas/mulmod.cpp}
			\subsection{Exponenciacion modular}
			Encuentra $(a^b)$ mod c, se nesecita implementar previamente multiplicacion modular.
			\cppfile[16-20]{matematicas/exp_modular.cpp}
			\subsection{Algoritmo extendido de euclides}
			Encuentra dos numeros x e y tal que: MCD(a, b) = ax + by
			\cppfile[5-24]{matematicas/algoritmo_extendido_de_euclides.cpp}
			\subsection{Inverso multiplicativo modular}
			Encuentra un x tal que (a * x) es congruente a 1 con modulo p,
				entonces:\\ (a * x) mod p = 1 mod p\\
				necesita del algoritmo extendido de euclides\\
			O(log m)
			\cppfile[17-27]{matematicas/inverso_multiplicativo_modular.cpp}
			\subsection{Phi de euler}
			Devuelve la cantidad de coprimos de un numero n\\
			O($\sqrt{n}$)
			\cppfile[5-16]{matematicas/phi_de_euler.cpp}
			\subsection{Rho de pollard}
			Factorizacion rapida, requiere de implementar previamente exponenciacion modular,
				multiplicacion modular y el MCD\\
			O($\sqrt[4]{n}$)
			\cppfile[26-89]{matematicas/rho_de_pollard.cpp}
			\subsection{BigInteger c++}
			\cppfile[1-161]{matematicas/biginteger.cpp}
		
		\section{Otros}
			\subsection{Busqueda binaria}
			O(log n)
			\cppfile[7-22]{otros/busqueda_binaria.cpp}
			\subsection{Raiz babilonica}
			Encuentra la raiz cuadrada de un numero
			\cppfile[4-12]{matematicas/raiz_babilonica.cpp}
			
		\section{Programacion dinamica}
			\subsection{Subconjuntos de un conjunto}
			O($2^{n}$)
			\cppfile[6-17]{programacion_dinamica/bitmask.cpp}
			\subsection{Problema de la mochila}
			\cppfile[8-23]{programacion_dinamica/knapsack.cpp}
			\subsection{Longest Increment Subsecuence}
			Subsecuencia creciente mas larga\\
			O(n log n)
			\cppfile[8-19]{programacion_dinamica/LIS.cpp}
			\subsection{Max Range Sum}
			O(n)
			\cppfile[6-22]{programacion_dinamica/Max_Range_Sum.cpp}
			\subsection{Subset Sum}
			\cppfile[8-20]{programacion_dinamica/Subset_Sum.cpp}
	
	%Tomado de notebook c++ de la ufps
		\section{Cadenas}
			\subsection{KMP}
			Encuentra si una cadena es subcadena de otra
			\cppfile[4-30]{cadenas/kmp.cpp}
	
	\newpage
	\section{Tips and formulas(ufps)}

\subsection{ASCII Table}
Caracteres ASCII con sus respectivos valores numéricos.


\begin{tabbing}
\textbf{No.}\hspace{1cm} \=  \textbf{ASCII}\hspace{2cm} \= \textbf{No.}\hspace{1cm} \= \textbf{ASCII}\hspace{2cm}  \\ 
0 \> NUL \> 16 \> DLE \\
1 \> SOH \> 17 \> DC1 \\
2 \> STX \> 18 \> DC2 \\
3 \> ETX \> 19 \> DC3 \\
4 \> EOT \> 20 \> DC4 \\
5 \> ENQ \> 21 \> NAK \\
6 \> ACK \> 22 \> SYN \\
7 \> BEL \> 23 \> ETB \\
8 \> BS \> 24 \> CAN \\
9 \> TAB \> 25 \> EM \\
10 \> LF \> 26 \> SUB \\
11 \> VT \> 27 \> ESC \\
12 \> FF \> 28 \> FS \\
13 \> CR \> 29 \> GS \\
14 \> SO \> 30 \> RS \\
15 \> SI \> 31 \> US \\ 
\end{tabbing}


\begin{tabbing}
\textbf{No.}\hspace{1cm} \=  \textbf{ASCII}\hspace{2cm} \= \textbf{No.}\hspace{1cm} \= \textbf{ASCII}\hspace{2cm}  \\ 
32 \> (space) \> 48 \> 0 \\
33 \> ! \> 49 \> 1 \\
34 \> " \> 50 \> 2 \\
35 \> \# \> 51 \> 3 \\
36 \> \$ \> 52 \> 4 \\
37 \> \% \> 53 \> 5 \\
38 \> \& \> 54 \> 6 \\
39 \> ' \> 55 \> 7 \\
40 \> ( \> 56 \> 8 \\
41 \> ) \> 57 \> 9 \\
42 \> * \> 58 \> : \\
43 \> + \> 59 \> ; \\
44 \> , \> 60 \> < \\
45 \> - \> 61 \> = \\
46 \> . \> 62 \> > \\
47 \> / \> 63 \> ? \\ 
\end{tabbing}

\begin{tabbing}
\textbf{No.}\hspace{1cm} \=  \textbf{ASCII}\hspace{2cm} \= \textbf{No.}\hspace{1cm} \= \textbf{ASCII}\hspace{2cm}  \\ 
64 \> @ \> 80 \> P \\
65 \> A \> 81 \> Q \\
66 \> B \> 82 \> R \\
67 \> C \> 83 \> S \\
68 \> D \> 84 \> T \\
69 \> E \> 85 \> U \\
70 \> F \> 86 \> V \\
71 \> G \> 87 \> W \\
72 \> H \> 88 \> X \\
73 \> I \> 89 \> Y \\
74 \> J \> 90 \> Z \\
75 \> K \> 91 \> [ \\
76 \> L \> 92 \> \textbackslash \\
77 \> M \> 93 \> ] \\
78 \> N \> 94 \> \textasciicircum \\
79 \> O \> 95 \> \_ \\ 
\end{tabbing}

\begin{tabbing}
\textbf{No.}\hspace{1cm} \=  \textbf{ASCII}\hspace{2cm} \= \textbf{No.}\hspace{1cm} \= \textbf{ASCII}\hspace{2cm}  \\ 
96 \> ` \> 112 \> p \\
97 \> a \> 113 \> q \\
98 \> b \> 114 \> r \\
99 \> c \> 115 \> s \\
100 \> d \> 116 \> t \\
101 \> e \> 117 \> u \\
102 \> f \> 118 \> v \\
103 \> g \> 119 \> w \\
104 \> h \> 120 \> x \\
105 \> i \> 121 \> y \\
106 \> j \> 122 \> z \\
107 \> k \> 123 \> \{ \\
108 \> l \> 124 \> \textbar \\
109 \> m \> 125 \> \} \\
110 \> n \> 126 \> \textasciitilde \\
111 \> o \> 127 \>  \\  
\end{tabbing}

\newpage
\subsection{Formulas}
\begin{center}
\tablefirsthead{}
\tabletail{
\midrule 
\multicolumn{2}{r}{{Continúa en la siguiente columna}} \\}
\tablelasttail{}
{\renewcommand{\arraystretch}{1.4}
\begin{supertabular}{|p{2.2cm}|p{8.2cm}|}
\hline
\multicolumn{2}{|c|}{} \\
\multicolumn{2}{|c|}{PERMUTACIÓN Y COMBINACIÓN} \\
\multicolumn{2}{|c|}{} \\ \hline
Combinación (Coeficiente Binomial) & Número de subconjuntos de k elementos escogidos de un conjunto con n elementos.

$ \binom{n}{k} = \binom{n}{n-k} = \displaystyle\frac{n!}{k!(n-k)!} $ 

\\ \hline

Combinación con repetición & Número de grupos formados por n elementos, partiendo de m tipos de elementos.

$ CR_{m}^{n} = \binom{m+n-1}{n} = \displaystyle\frac{(m + n - 1)!}{n!(m-1)!} $

\\ \hline
Permutación & Número de formas de agrupar n elementos, donde importa el orden y sin repetir elementos

$ P_{n} = n! $
\\ \hline
Permutación múltiple & 
Elegir r elementos de n posibles con repetición 


$ n^{r} $
\\ \hline
Permutación con repetición & Se tienen n elementos donde el primer elemento se repite a veces , el segundo b veces , el tercero c veces, ...

$ PR_{n}^{a,b,c...} = \displaystyle\frac{P_{n}}{a!b!c!...}$

\\ \hline
Permutaciones sin repetición & Núumero de formas de agrupar r elementos de n disponibles, sin repetir elementos


$\displaystyle\frac{n!}{(n-r)!}$

\\ \hline
\multicolumn{2}{|c|}{} \\
\multicolumn{2}{|c|}{DISTANCIAS} \\
\multicolumn{2}{|c|}{} \\ \hline
Distancia Euclideana & $d_{E}(P_{1},P_{2}) = \sqrt{(x_{2}-x_{1})^{2}+(y_{2}-y_{1})^{2}}$ \\ \hline
Distancia Manhattan & $d_{M}(P_{1}, P_{2}) = |x_{2} - x_{1}| + |y_{2} - y_{1}|$ \\ \hline
\multicolumn{2}{|c|}{} \\
\multicolumn{2}{|c|}{CIRCUNFERENCIA Y CÍRCULO} \\ 
\multicolumn{2}{|c|}{} \\ \hline
\multicolumn{2}{|p{12cm}|}{Considerando $r$ como el radio, $\alpha$ como el ángulo del arco o sector, y (R, r) como radio mayor y menor respectivamente.} \\ \hline
Área                   & $A = \pi * r^{2} $\\ \hline
Longitud               & $L = 2*\pi*r$  \\ \hline
Longitud de un arco    & $L = \displaystyle\frac{2*\pi*r*\alpha}{360}$  \\ \hline
Área sector circular   & $A = \displaystyle\frac{\pi * r^{2} * \alpha}{360}$ \\ \hline
Área corona circular   & $A = \pi  (R^{2} - r^{2})$ \\ \hline
\multicolumn{2}{|c|}{} \\
\multicolumn{2}{|c|}{TRIÁNGULO} \\ 
\multicolumn{2}{|c|}{} \\ \hline
\multicolumn{2}{|p{12cm}|}{Considerando $b$ como la longitud de la base, $h$ como la altura, letras minúsculas como la longitud de los lados, letras mayúsculas como los ángulos, y $r$ como el radio de círcunferencias asociadas.} \\ \hline
Área conociendo base y altura & $A = \displaystyle\frac{1}{2}b * h$ \\ \hline
Área conociendo 2 lados y el ángulo que forman & $A = \displaystyle\frac{1}{2}b*a*sin(C)$ \\ \hline
Área conociendo los 3 lados & $ A = \sqrt{p(p - a)(p - b)(p - c)}$ con $p = \displaystyle\frac{a + b + c}{2}$ \\ \hline
Área de un triángulo circunscrito a una circunferencia & $A = \displaystyle\frac{abc}{4r}$ \\ \hline
Área de un triángulo inscrito a una circunferencia & $A = r(\displaystyle\frac{a+b+c}{2})$ \\ \hline
Área de un triangulo equilátero & $A = \displaystyle\frac{\sqrt{3}}{4}a^{2}$ \\ \hline
\multicolumn{2}{|c|}{} \\
\multicolumn{2}{|c|}{RAZONES TRIGONOMÉTRICAS} \\
\multicolumn{2}{|c|}{} \\ \hline
\multicolumn{2}{|p{12cm}|}{Considerando un triangulo rectángulo de lados $a, b$ y $c$, con vértices $A, B$ y $C$ (cada vértice opuesto al lado cuya letra minuscula coincide con el) y un ángulo $\alpha$ con centro en el vertice $A$. a y b son catetos, c es la hipotenusa:}
\\ \hline
\multicolumn{2}{|p{12cm}|}{
$sin(\alpha) = \displaystyle\frac{cateto\ opuesto}{hipotenusa} = \displaystyle\frac{a}{c}$ 

} 
\\ \hline
\multicolumn{2}{|p{12cm}|}{
$cos(\alpha) = \displaystyle\frac{cateto\ adyacente}{hipotenusa} = \frac{b}{c}$ 

} 
\\ \hline
\multicolumn{2}{|p{12cm}|}{
$tan(\alpha) = \displaystyle\frac{cateto\ opuesto}{cateto\ adyacente} = \frac{a}{b}$ 

} 
\\ \hline
\multicolumn{2}{|p{12cm}|}{
$sec(\alpha) = \displaystyle\frac{1}{cos(\alpha)} = \frac{c}{b}$ 

}
\\ \hline
\multicolumn{2}{|p{12cm}|}{
$csc(\alpha) = \displaystyle\frac{1}{sin(\alpha)} = \frac{c}{a}$ 

} 
\\ \hline
\multicolumn{2}{|p{12cm}|}{
$cot(\alpha) = \displaystyle\frac{1}{tan(\alpha)} = \frac{b}{a}$ 

}
\\ \hline
\multicolumn{2}{|c|}{} \\
\multicolumn{2}{|c|}{PROPIEDADES DEL MÓDULO (RESIDUO)} \\
\multicolumn{2}{|c|}{} \\ \hline
Propiedad neutro & (a \% b) \% b = a \% b \\ \hline
Propiedad asociativa en multiplicación &  (ab) \% c = ((a \% c)(b \% c)) \% c \\ \hline
Propiedad asociativa en suma & (a + b) \% c = ((a \% c) + (b \% c)) \% c \\ \hline
\multicolumn{2}{|c|}{} \\
\multicolumn{2}{|c|}{CONSTANTES} \\
\multicolumn{2}{|c|}{} \\ \hline
Pi & $\pi = acos(-1) \approx 3.14159$ \\ \hline
e & $e \approx 2.71828$ \\ \hline
Número áureo & $\phi = \displaystyle\frac{1 + \sqrt{5}}{2} \approx 1.61803$ 

\\ \hline


\end{supertabular}
}
\end{center}
\subsection{Sequences}
Listado de secuencias mas comunes y como hallarlas.

\begin{center}
\tablefirsthead{}
\tabletail{
\midrule 
\multicolumn{2}{r}{{Continúa en la siguiente columna}} \\}
\tablelasttail{}
{\renewcommand{\arraystretch}{1.4}
\begin{supertabular}{|p{1.8cm}|p{8.6cm}|}

\hline

\multirow{2}{2cm}{Estrellas octangulares}
& 	0, 1, 14, 51, 124, 245, 426, 679, 1016, 1449, 1990, 2651, ...
\\ \cline{2-2}
& $f(n) = n*(2*n^{2} - 1)$.
\\ \hline


\multirow{2}{2cm}
{Euler totient}    
& 1, 1, 2, 2, 4, 2, 6, 4, 6, 4, 10, 4, 12, 6,...            
\\ \cline{2-2} 
& $f(n) = $ Cantidad de números naturales $\leq n$ coprimos con n. 
\\ \hline

\multirow{2}{2cm}{Números de Bell} 
& 1, 1, 2, 5, 15, 52, 203, 877, 4140, 21147, 115975, ...
\\ \cline{2-2} 
& Se inicia una matriz triangular con f[0][0] = f[1][0] = 1. La suma de estos dos se guarda en f[1][1] y se traslada a f[2][0]. Ahora se suman f[1][0] con f[2][0] y se guarda en f[2][1]. Luego se suman f[1][1] con f[2][1] y se guarda en f[2][2] trasladandose a f[3][0] y así sucesivamente. Los valores de la primera columna contienen la respuesta.
\\ \hline


\multirow{2}{2cm}
{Números de Catalán} 
& 1, 1, 2, 5, 14, 42, 132, 429, 1430, 4862, 16796, 58786, ...
\\ \cline{2-2}
& $f(n)=\displaystyle\frac{(2n)!}{(n + 1)! n!}$

\\ \hline

\multirow{2}{2cm}{Números de Fermat}
& 3, 5, 17, 257, 65537, 4294967297, 18446744073709551617, ...
\\ \cline{2-2}
& $f(n) = 2^{(\displaystyle2^{\textstyle n})} + 1$
\\ \hline


\multirow{2}{2cm}
{Números de Fibonacci} 
& 0, 1, 1, 2, 3, 5, 8, 13, 21, 34, 55, 89, 144, 233, ...    
\\ \cline{2-2} 
& $f(0) = 0$; $f(1) = 1$; $f(n) = f(n-1) + f(n-2)$ para $n>1$             \\ \hline

\multirow{2}{2cm}
{Números de Lucas} 
& 2, 1, 3, 4, 7, 11, 18, 29, 47, 76, 123, 199, 322, ...    
\\ \cline{2-2} 
& $f(0) = 2$; $f(1) = 1$; $f(n) = f(n-1) + f(n-2)$ para $n>1$            
\\ \hline

\multirow{2}{2cm}{Números de Pell} 
& 0, 1, 2, 5, 12, 29, 70, 169, 408, 985, 2378, 5741, 13860, ...
\\ \cline{2-2} 
& $f(0) = 0; f(1) = 1; f(n) = 2f(n-1) + f(n-2)$ para $n>1$
\\ \hline

\multirow{2}{2cm}
{Números de Tribonacci} 
& 0, 0, 1, 1, 2, 4, 7, 13, 24, 44, 81, 149, 274, 504, ...    
\\ \cline{2-2} 
& $f(0)=f(1)=0; f(2)=1; f(n) = f(n-1) + f(n-2) + f(n-3)$ para $n>2$
\\ \hline

\multirow{2}{2cm}{Números factoriales}
& 1, 1, 2, 6, 24, 120, 720, 5040, 40320, 362880, ...
\\ \cline{2-2}
&$ f(0) = 1; f(n) = \displaystyle\prod_{\textstyle k=1}^{\textstyle n}k$ para $n>0$.

\\ \hline

\multirow{2}{2cm}{Números piramidales cuadrados}
& 0, 1, 5, 14, 30, 55, 91, 140, 204, 285, 385, 506, 650, ...
\\ \cline{2-2}
& $f(n) = \displaystyle\frac{n*(n+1)*(2*n+1)}{6}$

\\ \hline

\multirow{2}{2cm}{Números primos de Mersenne}
& 3, 7, 31, 127, 8191, 131071, 524287, 2147483647, ...
\\ \cline{2-2}
& $f(n) = 2^{p(n)} - 1$ donde $p$ representa valores primos iniciando en $p(0)=2$.
\\ \hline


\multirow{2}{2cm}{Números tetraedrales}
& 0, 1, 3, 6, 10, 15, 21, 28, 36, 45, 55, 66, 78, 91, 105, ...
\\ \cline{2-2}
& $f(n) = \displaystyle\frac{n*(n+1)*(n+2)}{6}$

\\ \hline


\multirow{2}{2cm}{Números triangulares}
& 0, 1, 3, 6, 10, 15, 21, 28, 36, 45, 55, 66, 78, 91, 105, ...
\\ \cline{2-2}
& $f(n) = \displaystyle\frac{n(n+1)}{2}$

\\ \hline


\multirow{2}{2cm}{OEIS A000127}
& 1, 2, 4, 8, 16, 31, 57, 99, 163, 256, 386, 562, ...
\\ \cline{2-2}
& $f(n) = \displaystyle\frac{(n^{4}-6n^{3}+23n^{2}-18{n}+24)}{24}$.

\\ \hline


\multirow{2}{2cm}{Secuencia de Narayana}
& 1, 1, 1, 2, 3, 4, 6, 9, 13, 19, 28, 41, 60, 88, 129, ...
\\ \cline{2-2}
& $f(0) = f(1) = f(2) = 1; f(n) = f(n-1) + f(n-3)$ para todo $n>2$.
\\ \hline


\multirow{2}{2cm}
{Secuencia de Silvestre} 
& 2, 3, 7, 43, 1807, 3263443, 10650056950807, ...    
\\ \cline{2-2} 
& $f(0) = 2; f(n+1) = f(n)^2 - f(n) + 1$               
\\ \hline

\multirow{2}{2cm}{Secuencia de vendedor perezoso} 
& 1, 2, 4, 7, 11, 16, 22, 29, 37, 46, 56, 67, 79, 92, 106, ...
\\ \cline{2-2} 
& Equivale al triangular(n) + 1. Máxima número de piezas que se pueden formar al hacer n cortes a un disco. 

$f(n) = \displaystyle\frac{n(n+1)}{2} + 1$

\\ \hline

\multirow{2}{2cm}{Suma de los divisores de un número}
& 1, 3, 4, 7, 6, 12, 8, 15, 13, 18, 12, 28, 14, 24, ...
\\ \cline{2-2}
&Para todo $n>1$ cuya descomposición en factores primos es $n=\displaystyle p_{1}^{\textstyle a_{1}}\displaystyle p_{2}^{\textstyle a_{2}}...\displaystyle p_{k}^{\textstyle a_{k}}$ se tiene que:


$f(n) = \displaystyle\frac{p_{1}^{a_{1} + 1} - 1}{p_{1} - 1} * \frac{p_{2}^{a_{2} + 1} - 1}{p_{2} - 1} * ... * \frac{p_{k}^{a_{k} + 1} - 1}{p_{k} - 1}$ 

\\ \hline
\end{supertabular}
}
\end{center}
\subsection{Time Complexities}

Aproximación del mayor número n de datos que pueden procesarse para cada una de las complejidades algoritmicas. Tomar esta tabla solo como referencia.

\begin{tabbing}
\textbf{Complexity}\hspace{4cm} \=  \textbf{n}\hspace{3cm}   \\ 
$O(n!)$ \> 11\\ 
$O(n^{5})$ \> 50\\ 
$O(2^{n}*n^{2})$ \> 18\\ 
$O(2^{n}*n)$ \> 22\\ 
$O(n^{4})$ \> 100\\ 
$O(n^{3})$ \> 500\\ 
$O(n^{2}\log_{2}n)$ \> 1.000\\ 
$O(n^{2})$ \> 10.000\\ 
$O(n\log_{2}n)$ \> $10^{6}$\\ 
$O(n)$ \> $10^{8}$\\ 
$O(\sqrt{n})$ \> $10^{16}$\\ 
$O(\log_{2}n)$ \> -\\ 
$O(1)$ \> -\\ 
\end{tabbing}

	\section{Extras}
		\subsection{Formulas extra}
			\textbf{formula de triangulos degenerados:}\\
			\vspace{3mm}
			{\Large$\frac{(a + b - c) * (a + c - b) * (b  + c - a)} {a * b * c}$}
			\vspace{3mm}
			\\Si el resultado es mayor que 0.5, es posible formar el triangulo.
	
			\vspace{8mm}
			\textbf{Ecuacion de la recta que pasa por dos puntos:}\\
			\vspace{3mm}
			{\Large $\frac{x - x_{1}}{x_{2} - x_{1}} = \frac{y - y_{1}}{y_{2} - y_{1}}$}

			\vspace{8mm}
			\textbf{Distancia de un punto a una recta:}\\
			Teniendo una recta con formula de la forma: $ax + by + c$ la distancia minima
			a un punto p de la forma $(x, y)$ la distancia minima esta dada por la formula:\\
			\vspace{3mm}
			{\Large $d = \frac{ax + by + c}{\sqrt{a^{2} + b^{2}}}$}
			\vspace{3mm}
			
			\vspace{8mm}
			\textbf{Formula de numeros fibonacci:}\\
			\vspace{3mm}
			{\Large $f(n) = \frac{1}{\sqrt{5}} * [ (\frac{1 + \sqrt{5}}{2})^{2}  - (\frac{1 - \sqrt{5}}{2})^{2}]$ }
			
		\newpage		
		\subsection{Secuencias}
		
		\textbf{Primos:}\\
		\vspace{3mm}
			2 3 5 7 11 13 17 19 23 29 31 37 41 43 47 53 59 61 67 71 73 79 83 89 97 101 103 107 109 113 127 131 137 139 149
			151 157 163 167 173 179 181 191 193 197 199 211 223 227 229 233 239 241 251 257 263 269 271 277 281 283 293 307
			311 313 317 331 337 347 349 353 359 367 373 379 383 389 397 401 409 419 421 431 433 439 443 449 457 461 463 467
			479 487 491 499 503 509 521 523 541 547 557 563 569 571 577 587 593 599 601 607 613 617 619 631 641 643 647 653
			659 661 673 677 683 691 701 709 719 727 733 739 743 751 757 761 769 773 787 797 809 811 821 823 827 829 839 853
			857 859 863 877 881 883 887 907 911 919 929 937 941 947 953 967 971 977 983 991 997 1009 1013 1019 1021 1031
			1033 1039 1049 1051 1061 1063 1069 1087 1091 1093 1097 1103 1109 1117 1123 1129 1151 1153 1163 1171 1181 1187
			1193 1201 1213 1217 1223 1229 1231 1237 1249 1259 1277 1279 1283 1289 1291 1297 1301 1303 1307 1319 1321 1327
			1361 1367 1373 1381 1399 1409 1423 1427 1429 1433 1439 1447 1451 1453 1459 1471 1481 1483 1487 1489 1493 1499
			1511 1523 1531 1543 1549 1553 1559 1567 1571 1579 1583 1597 1601 1607 1609 1613 1619 1621 1627 1637 1657 1663
			1667 1669 1693 1697 1699 1709 1721 1723 1733 1741 1747 1753 1759 1777 1783 1787 1789 1801 1811 1823 1831 1847
			1861 1867 1871 1873 1877 1879 1889 1901 1907 1913 1931 1933 1949 1951 1973 1979 1987 1993 1997 1999 2003 2011
			2017 2027 2029 2039 2053 2063 2069 2081 2083 2087 2089 2099 2111 2113 2129 2131 2137 2141 2143 2153 2161 2179
			2203 2207 2213 2221 2237 2239 2243 2251 2267 2269 2273 2281 2287 2293 2297 2309 2311 2333 2339 2341 2347 2351
			2357 2371 2377 2381 2383 2389 2393 2399 2411 2417 2423 2437 2441 2447 2459 2467 2473 2477 2503 2521 2531 2539
			2543 2549 2551 2557 2579 2591 2593 2609 2617 2621 2633 2647 2657 2659 2663 2671 2677 2683 2687 2689 2693 2699
			2707 2711 2713 2719 2729 2731 2741 2749 2753 2767 2777 2789 2791 2797 2801 2803 2819 2833 2837 2843 2851 2857
			2861 2879 2887 2897 2903 2909 2917 2927
	
	\vspace{8mm}
	\textbf{Fibonacci:}\\
	\vspace{3mm}
	0 1 1 2 3 5 8 13 21 34 55 89 144 233 377 610 987 1597 2584 4181 6765 10946 17711 28657 46368 75025 121393 196418 317811
	514229 832040 1346269 2178309 3524578 5702887 9227465 14930352 24157817 39088169 63245986 102334155 165580141 267914296
	433494437 701408733 1134903170 1836311903
	
	\vspace{8mm}
	\textbf{Factoriales:}\\
	\vspace{3mm}
	1 2 6 24 120 720 5040 40320 362880 3628800 39916800 479001600 6227020800 87178291200 1307674368000 20922789888000
	355687428096000 6402373705728000 121645100408832000
	
	\vspace{8mm}
	\textbf{Potencias de dos:} de 1 hasta 63\\
	\vspace{3mm}
	1 2 4 8 16 32 64 128 256 512 1024 2048 4096 8192 16384 32768 65536 131072 262144 524288 1048576 2097152 4194304 8388608
	16777216 33554432 67108864 134217728 268435456 536870912 1073741824 2147483648 4294967296 8589934592 17179869184
	34359738368 68719476736 137438953472 274877906944 549755813888 1099511627776 2199023255552 4398046511104 8796093022208
	17592186044416 35184372088832 70368744177664 140737488355328 281474976710656 562949953421312 1125899906842624
	2251799813685248 4503599627370496 9007199254740992 18014398509481984 36028797018963968 72057594037927936
	144115188075855872 288230376151711744 576460752303423488 1152921504606846976 2305843009213693952 4611686018427387904
	9223372036854775808
\end{document}


